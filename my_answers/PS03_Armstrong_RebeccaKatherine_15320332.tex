\documentclass[12pt,letterpaper]{article}
\usepackage{graphicx,textcomp}
\usepackage{natbib}
\usepackage{setspace}
\usepackage{fullpage}
\usepackage{color}
\usepackage[reqno]{amsmath}
\usepackage{amsthm}
\usepackage{fancyvrb}
\usepackage{amssymb,enumerate}
\usepackage[all]{xy}
\usepackage{endnotes}
\usepackage{lscape}
\newtheorem{com}{Comment}
\usepackage{float}
\usepackage{hyperref}
\newtheorem{lem} {Lemma}
\newtheorem{prop}{Proposition}
\newtheorem{thm}{Theorem}
\newtheorem{defn}{Definition}
\newtheorem{cor}{Corollary}
\newtheorem{obs}{Observation}
\usepackage[compact]{titlesec}
\usepackage{dcolumn}
\usepackage{tikz}
\usetikzlibrary{arrows}
\usepackage{multirow}
\usepackage{xcolor}
\newcolumntype{.}{D{.}{.}{-1}}
\newcolumntype{d}[1]{D{.}{.}{#1}}
\definecolor{light-gray}{gray}{0.65}
\usepackage{url}
\usepackage{listings}
\usepackage{color}

\definecolor{codegreen}{rgb}{0,0.6,0}
\definecolor{codegray}{rgb}{0.5,0.5,0.5}
\definecolor{codepurple}{rgb}{0.58,0,0.82}
\definecolor{backcolour}{rgb}{0.95,0.95,0.92}

\lstdefinestyle{mystyle}{
	backgroundcolor=\color{backcolour},   
	commentstyle=\color{codegreen},
	keywordstyle=\color{magenta},
	numberstyle=\tiny\color{codegray},
	stringstyle=\color{codepurple},
	basicstyle=\footnotesize,
	breakatwhitespace=false,         
	breaklines=true,                 
	captionpos=b,                    
	keepspaces=true,                 
	numbers=left,                    
	numbersep=5pt,                  
	showspaces=false,                
	showstringspaces=false,
	showtabs=false,                  
	tabsize=2
}
\lstset{style=mystyle}
\newcommand{\Sref}[1]{Section~\ref{#1}}
\newtheorem{hyp}{Hypothesis}

\title{Problem Set 3}
\date{Due: November 11, 2024}
\author{Applied Stats/Quant Methods 1}


\begin{document}
	\maketitle
	\section*{Instructions}
	\begin{itemize}
		\item Please show your work! You may lose points by simply writing in the answer. If the problem requires you to execute commands in \texttt{R}, please include the code you used to get your answers. Please also include the \texttt{.R} file that contains your code. If you are not sure if work needs to be shown for a particular problem, please ask.
	\item Your homework should be submitted electronically on GitHub.
	\item This problem set is due before 23:59 on Sunday November 11, 2024. No late assignments will be accepted.

	\end{itemize}

		\vspace{.25cm}
	
\noindent In this problem set, you will run several regressions and create an add variable plot (see the lecture slides) in \texttt{R} using the \texttt{incumbents\_subset.csv} dataset. Include all of your code.

	\vspace{.5cm}
\section*{Question 1}
\vspace{.25cm}
\noindent We are interested in knowing how the difference in campaign spending between incumbent and challenger affects the incumbent's vote share. 
	\begin{enumerate}
		\item Run a regression where the outcome variable is \texttt{voteshare} and the explanatory variable is \texttt{difflog}.	\vspace{0.1cm}
		\lstinputlisting[language=R, firstline=34, lastline=35]{C:/Users/isabel/Documents/GitHub/StatsI_Fall2024/my_answers/PS03_Armstrong_RebeccaKatherine_15320332.R} 
		\vspace{7cm}
		\begin{table}[!htbp] \centering 
			\caption{Regression Summary: Outcome VoteShare, Explanatory: DiffLog} 
			\label{} 
			\begin{tabular}{@{\extracolsep{5pt}}lc} 
				\\[-1.8ex]\hline 
				\hline \\[-1.8ex] 
				& \multicolumn{1}{c}{\textit{Dependent variable:}} \\ 
				\cline{2-2} 
				\\[-1.8ex] & difflog \\ 
				\hline \\[-1.8ex] 
				voteshare & 8.816$^{***}$ \\ 
				& (0.205) \\ 
				& \\ 
				Constant & $-$3.949$^{***}$ \\ 
				& (0.136) \\ 
				& \\ 
				\hline \\[-1.8ex] 
				Observations & 3,193 \\ 
				R$^{2}$ & 0.367 \\ 
				Adjusted R$^{2}$ & 0.367 \\ 
				Residual Std. Error & 1.144 (df = 3191) \\ 
				F Statistic & 1,852.791$^{***}$ (df = 1; 3191) \\ 
				\hline 
				\hline \\[-1.8ex] 
				\textit{Note:}  & \multicolumn{1}{r}{$^{*}$p$<$0.1; $^{**}$p$<$0.05; $^{***}$p$<$0.01} \\ 
			\end{tabular} 
		\end{table} 
		\vspace{7cm}
		\item Make a scatterplot of the two variables and add the regression line. 	\vspace{0.1cm}
		\lstinputlisting[language=R, firstline=38, lastline=39]{C:/Users/isabel/Documents/GitHub/StatsI_Fall2024/my_answers/PS03_Armstrong_RebeccaKatherine_15320332.R}  
		\vspace{0.1cm}
		\includegraphics[width=0.5\textwidth]{C:/Users/isabel/Documents/GitHub/StatsI_Fall2024/problemSets/PS03/ps03plot1.jpg}
		\vspace{0.1cm}
		\item Save the residuals of the model in a separate object.	\vspace{0.1cm}
		\lstinputlisting[language=R, firstline=41, lastline=42]{C:/Users/isabel/Documents/GitHub/StatsI_Fall2024/my_answers/PS03_Armstrong_RebeccaKatherine_15320332.R}  
		\begin{verbatim}
			The residuals of the model have been saved as an object named 
			'res_reg' in R.  As the object is too large to add to the latex 
			pdf, a sample of the head of it is below:
		\end{verbatim}
		\vspace{0.1cm}
		\input{C:/Users/isabel/Documents/GitHub/StatsI_Fall2024/problemSets/PS03/ps03q1res.tex}
		\item Write the prediction equation.  
		\vspace{0.1cm}
		\text Prediction equation for this regression is:
		\[
		\text{voteshare} = \beta_0 + \beta_1 \cdot \text{difflog} + \epsilon
		\]
		\text where voteshare is the outcome variable, and difflog is the explanatory variable.
		\vspace{5cm}
	\end{enumerate}
	
\newpage

\section*{Question 2}
\noindent We are interested in knowing how the difference between incumbent and challenger's spending and the vote share of the presidential candidate of the incumbent's party are related.	\vspace{.25cm}
	\begin{enumerate}
		\item Run a regression where the outcome variable is \texttt{presvote} and the explanatory variable is \texttt{difflog}.	\vspace{0.1cm}
		\lstinputlisting[language=R, firstline=51, lastline=52]{C:/Users/isabel/Documents/GitHub/StatsI_Fall2024/my_answers/PS03_Armstrong_RebeccaKatherine_15320332.R} 
		\vspace{5cm}
		\begin{table}[!htbp] \centering 
			\caption{Regression Summary: Outcome Presvote, Explanatory: DiffLog} 
			\label{} 
			\begin{tabular}{@{\extracolsep{5pt}}lc} 
				\\[-1.8ex]\hline 
				\hline \\[-1.8ex] 
				& \multicolumn{1}{c}{\textit{Dependent variable:}} \\ 
				\cline{2-2} 
				\\[-1.8ex] & difflog \\ 
				\hline \\[-1.8ex] 
				presvote & 3.690$^{***}$ \\ 
				& (0.210) \\ 
				& \\ 
				Constant & $-$0.206$^{*}$ \\ 
				& (0.118) \\ 
				& \\ 
				\hline \\[-1.8ex] 
				Observations & 3,193 \\ 
				R$^{2}$ & 0.088 \\ 
				Adjusted R$^{2}$ & 0.088 \\ 
				Residual Std. Error & 1.374 (df = 3191) \\ 
				F Statistic & 307.715$^{***}$ (df = 1; 3191) \\ 
				\hline 
				\hline \\[-1.8ex] 
				\textit{Note:}  & \multicolumn{1}{r}{$^{*}$p$<$0.1; $^{**}$p$<$0.05; $^{***}$p$<$0.01} \\ 
			\end{tabular} 
		\end{table}  \vspace{5cm}
		\item Make a scatterplot of the two variables and add the regression line. 	
		\lstinputlisting[language=R, firstline=56, lastline=57]{C:/Users/isabel/Documents/GitHub/StatsI_Fall2024/my_answers/PS03_Armstrong_RebeccaKatherine_15320332.R} 
		\includegraphics[width=0.5\textwidth]{C:/Users/isabel/Documents/GitHub/StatsI_Fall2024/problemSets/PS03/ps03plot2.jpg}
		\vspace{0.1cm}
		\item Save the residuals of the model in a separate object.	\vspace{0.1cm}
		\lstinputlisting[language=R, firstline=59, lastline=60]{C:/Users/isabel/Documents/GitHub/StatsI_Fall2024/my_answers/PS03_Armstrong_RebeccaKatherine_15320332.R} 
		\begin{verbatim}
			Similar to the previous question, the object which now holds the residuals 
			(here, 'reg_res2') is too large to display in it's entirety in this 
			pdf, and so the head of it is displayed below:
		\end{verbatim}
		\input{C:/Users/isabel/Documents/GitHub/StatsI_Fall2024/problemSets/PS03/ps03q2res.tex}
		\vspace{0.1cm}
		\item Write the prediction equation.
		\text The prediction equation for this regression is:
		\[
		\text{Presvote} = \beta_0 + \beta_1 \cdot \text{DiffLog} + \epsilon
		\]
		with the explanatory variable being difflog and the outcome variable being presvote.
		\vspace{5cm}
	\end{enumerate}
	
	\newpage	
\section*{Question 3}

\noindent We are interested in knowing how the vote share of the presidential candidate of the incumbent's party is associated with the incumbent's electoral success.
	\vspace{.25cm}
	\begin{enumerate}
		\item Run a regression where the outcome variable is \texttt{voteshare} and the explanatory variable is \texttt{presvote}.
		\vspace{0.1cm}
		\lstinputlisting[language=R, firstline=66, lastline=68]{C:/Users/isabel/Documents/GitHub/StatsI_Fall2024/my_answers/PS03_Armstrong_RebeccaKatherine_15320332.R}
		\begin{table}[!htbp] \centering 
			\caption{Regression Summary: Outcome VoteShare, Explanatory Presvote} 
			\label{} 
			\begin{tabular}{@{\extracolsep{5pt}}lc} 
				\\[-1.8ex]\hline 
				\hline \\[-1.8ex] 
				& \multicolumn{1}{c}{\textit{Dependent variable:}} \\ 
				\cline{2-2} 
				\\[-1.8ex] & presvote \\ 
				\hline \\[-1.8ex] 
				voteshare & 0.530$^{***}$ \\ 
				& (0.018) \\ 
				& \\ 
				Constant & 0.204$^{***}$ \\ 
				& (0.012) \\ 
				& \\ 
				\hline \\[-1.8ex] 
				Observations & 3,193 \\ 
				R$^{2}$ & 0.206 \\ 
				Adjusted R$^{2}$ & 0.206 \\ 
				Residual Std. Error & 0.103 (df = 3191) \\ 
				F Statistic & 826.950$^{***}$ (df = 1; 3191) \\ 
				\hline 
				\hline \\[-1.8ex] 
				\textit{Note:}  & \multicolumn{1}{r}{$^{*}$p$<$0.1; $^{**}$p$<$0.05; $^{***}$p$<$0.01} \\ 
			\end{tabular} 
		\end{table} 
		 \vspace{5cm}
		\item Make a scatterplot of the two variables and add the regression line. 
		\vspace{0.1cm}
		\lstinputlisting[language=R, firstline=70, lastline=71
		]{C:/Users/isabel/Documents/GitHub/StatsI_Fall2024/my_answers/PS03_Armstrong_RebeccaKatherine_15320332.R} 
		\includegraphics[width=0.5\textwidth]{C:/Users/isabel/Documents/GitHub/StatsI_Fall2024/problemSets/PS03/ps03plot3.jpg}
		\vspace{0.1cm}
		\item Write the prediction equation.
		\text Prediction equation for this regression is:
		\[
		\text{voteshare} = \beta_0 + \beta_1 \cdot \text{presvote} + \epsilon
		\]
		\vspace{5cm}
		\text where voteshare is the outcome variable, and presvote is the explanatory variable.
	\end{enumerate}
	

\newpage	
\section*{Question 4}
\noindent The residuals from part (a) tell us how much of the variation in \texttt{voteshare} is $not$ explained by the difference in spending between incumbent and challenger. The residuals in part (b) tell us how much of the variation in \texttt{presvote} is $not$ explained by the difference in spending between incumbent and challenger in the district.
	\begin{enumerate}
		\item Run a regression where the outcome variable is the residuals from Question 1 and the explanatory variable is the residuals from Question 2.	\vspace{0.1cm}
		\lstinputlisting[language=R, firstline=79, lastline=80]{C:/Users/isabel/Documents/GitHub/StatsI_Fall2024/my_answers/PS03_Armstrong_RebeccaKatherine_15320332.R} 
		\begin{table}[!htbp] \centering 
			\caption{Regression Summary: Outcome Q1 Residuals, Explanatory Q2 Residuals} 
			\label{} 
			\begin{tabular}{@{\extracolsep{5pt}}lc} 
				\\[-1.8ex]\hline 
				\hline \\[-1.8ex] 
				& \multicolumn{1}{c}{\textit{Dependent variable:}} \\ 
				\cline{2-2} 
				\\[-1.8ex] & res\_reg2 \\ 
				\hline \\[-1.8ex] 
				res\_reg & 0.990$^{***}$ \\ 
				& (0.012) \\ 
				& \\ 
				Constant & $-$0.000 \\ 
				& (0.014) \\ 
				& \\ 
				\hline \\[-1.8ex] 
				Observations & 3,193 \\ 
				R$^{2}$ & 0.680 \\ 
				Adjusted R$^{2}$ & 0.680 \\ 
				Residual Std. Error & 0.778 (df = 3191) \\ 
				F Statistic & 6,771.208$^{***}$ (df = 1; 3191) \\ 
				\hline 
				\hline \\[-1.8ex] 
				\textit{Note:}  & \multicolumn{1}{r}{$^{*}$p$<$0.1; $^{**}$p$<$0.05; $^{***}$p$<$0.01} \\ 
			\end{tabular} 
		\end{table} 
		\vspace{0.1cm}
		\item Make a scatterplot of the two residuals and add the regression line. 	\vspace{0.1cm}
		\lstinputlisting[language=R, firstline=83, lastline=84]{C:/Users/isabel/Documents/GitHub/StatsI_Fall2024/my_answers/PS03_Armstrong_RebeccaKatherine_15320332.R} 
		\includegraphics[width=0.5\textwidth]{C:/Users/isabel/Documents/GitHub/StatsI_Fall2024/problemSets/PS03/ps03plot4.jpg}
		\vspace{0.1cm}
		\item Write the prediction equation.
		\text{Prediction equation for this regression is:} \\
		\[
		\text{res\_reg} = \beta_0 + \beta_1 \cdot \text{res\_reg2} + \epsilon
		\]
		\vspace{0.5cm}
		\begin{verbatim}
			where the object 'res_reg' (the residuals from the first regression) is the 
			outcome variable, and object 'res_reg2' (the residuals from the second 
			regression) is the explanatory variable.}
		\end{verbatim}
		\vspace{5cm}
	\end{enumerate}
	
	\newpage	

\section*{Question 5}
\noindent What if the incumbent's vote share is affected by both the president's popularity and the difference in spending between incumbent and challenger? 
	\begin{enumerate}
		\item Run a regression where the outcome variable is the incumbent's \texttt{voteshare} and the explanatory variables are \texttt{difflog} and \texttt{presvote}.	\vspace{0.1cm}
		\lstinputlisting[language=R, firstline=89, lastline=89]{C:/Users/isabel/Documents/GitHub/StatsI_Fall2024/my_answers/PS03_Armstrong_RebeccaKatherine_15320332.R} 
		\begin{table}[!htbp] \centering 
			\caption{Regression Summary: Outcome Voteshare, Explanatory difflog and presvote} 
			\label{} 
			\begin{tabular}{@{\extracolsep{5pt}}lc} 
				\\[-1.8ex]\hline 
				\hline \\[-1.8ex] 
				& \multicolumn{1}{c}{\textit{Dependent variable:}} \\ 
				\cline{2-2} 
				\\[-1.8ex] & voteshare \\ 
				\hline \\[-1.8ex] 
				difflog & 0.036$^{***}$ \\ 
				& (0.001) \\ 
				& \\ 
				presvote & 0.257$^{***}$ \\ 
				& (0.012) \\ 
				& \\ 
				Constant & 0.449$^{***}$ \\ 
				& (0.006) \\ 
				& \\ 
				\hline \\[-1.8ex] 
				Observations & 3,193 \\ 
				R$^{2}$ & 0.450 \\ 
				Adjusted R$^{2}$ & 0.449 \\ 
				Residual Std. Error & 0.073 (df = 3190) \\ 
				F Statistic & 1,302.947$^{***}$ (df = 2; 3190) \\ 
				\hline 
				\hline \\[-1.8ex] 
				\textit{Note:}  & \multicolumn{1}{r}{$^{*}$p$<$0.1; $^{**}$p$<$0.05; $^{***}$p$<$0.01} \\ 
			\end{tabular} 
		\end{table} 
		\vspace{0.1cm}
		\item Write the prediction equation.	\vspace{0.1cm}
		\text Prediction equation for this regression is:
		\[
		\text{voteshare} = \beta_0 + \beta_1 \cdot \text{difflog} + \beta_2 \cdot \text{presvote} + \epsilon
		\]
		\vspace{5cm}
		\text where voteshare is the outcome variable, and difflog and presvote are the explanatory variables.
		\item What is it in this output that is identical to the output in Question 4? Why do you think this is the case?
		\begin{verbatim}
			The only things which both regressions have in common are 
			the number of observations - as both utilise the same dataset.  
			Despite the overlap of variables across both, they are modelling
			different relationships between these variables and so the coefficients, 
			standard errors, etc. are not maintained across both.
		\end{verbatim} 
		\vspace{5cm}
	\end{enumerate}




\end{document}
